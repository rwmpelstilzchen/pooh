\documentclass[twoside, 10pt]{book}
\usepackage[a5paper]{geometry}

\usepackage[nooverlap]{ruby}
%\renewcommand{\rubysep}{-0.35cm}
\renewcommand{\rubysize}{0.8}
\newcommand{\transliterate}[2]{\ruby{\H{#1}}{\LR{\grey{\trans{#2}}}}}
\newcommand{\transcribe}[2]{\ruby{\H{#1}}{\LR{\grey{\trans{#2}}}}}
\newcommand{\translate}[2]{\H{#1}~\grey{‘#2’}}
\newcommand{\sofit}[1]{\textsc{#1}}

\usepackage{xcolor}
\newcommand{\grey}[1]{\textcolor{gray}{#1}}

\usepackage{environ}
\usepackage{longtable}


\newcounter{paragraphno}
\setcounter{paragraphno}{0}
\newcommand{\newparagraph}{\stepcounter{paragraphno}\setcounter{sentenceno}{0}}
\newcounter{sentenceno}
\setcounter{sentenceno}{0}

\usepackage{titlesec}
%\titleformat{\section}[leftmargin]
%{\normalfont
%	\titlerule*[.6em]{\bfseries.}%
%	\vspace{6pt}%
%	}
%	{\thesection}{.5em}{}
%\titlespacing{\section}
%{2pc}{1.5ex plus .1ex minus .2ex}{0pc}
\newcommand{\sectionfont}{\fontspec{Fontin}}
\titleformat{\section}{\sectionfont\Large}{\textbf{{\thesection}}\sectionornament}{0.0cm}{}[\titlerule]
\titleformat{\subsection}{\sectionfont\large}{\textbf{{\thesubsection}}\sectionornament}{0.0cm}{}[\titleline{\color{gray}\titlerule}] % Solution from http://tex.stackexchange.com/questions/40088/strange-behaviour-of-titlesec-with-colored-titlerule



%\renewcommand{\labelitemi}{\symbolglyph{▷}}
%\renewcommand{\labelitemii}{\symbolglyph{▻}}
%\renewcommand{\labelitemii}{}
%\renewcommand{\labelitemiii}{\symbolglyph{▹}}
\NewEnviron{gloss}[2]
%{\begin{description}\begin{desclistitemenv}{\ruby{\H{#1}}{\grey{#2}}}}
%			{\end{desclistitemenv}\end{description}}
{
	\begin{description}
		\item \ruby{\H{\addfontfeature{FakeBold=0}#1}}{\grey{#2}}\hspace{1em}\grey{\symbolglyph{⁘}}\hspace{1em}\BODY
	\end{description}
}
\newcommand{\glsec}[1]{{\fontspec{Vesper Pro}\textsc{#1.}}\hspace{1em}}

\usepackage{marginnote}
\usepackage{framed}
\NewEnviron{sentence}[3]{
	\stepcounter{sentenceno}

	\begin{longtable}{p{0.48\linewidth}p{0.48\linewidth}}
		\marginnote{\begin{minipage}{1.5cm}{\begin{center}\color{gray}{\fontspec{Vesper Pro}\textsc{Pooh}}\\\arabic{paragraphno}:\arabic{sentenceno}\end{center}}\end{minipage}}%
		\trans{#2} & \RL{\H{#1\hfill~}}\\[1em]
		\multicolumn{2}{p{1\linewidth}}{#3}
	\end{longtable}

	\BODY

	\vspace{0.5cm}
	\begin{center}***\end{center}
	\vspace{0.5cm}
}



\usepackage{xltxtra}
\usepackage{fontspec}
\usepackage{bidi}
\setmainfont{Vesper Pro}
\renewcommand{\H}[1]{\RL{\fontspec[Script=Hebrew]{Rutz_OE Regular Pro}#1}}
\newcommand{\trans}[1]{\LR{\fontspec{Vesper Pro}\textit{\addfontfeature{HyphenChar=None}#1}}}
\newcommand{\gp}[1]{\LR{\fontspec{Vesper Pro}#1}}
\newcommand{\IPA}[1]{{\fontspec{Gentium}#1}}
\newcommand{\hl}[1]{\textbf{#1}}
\newcommand{\symbolglyph}[1]{{\fontspec{Symbola}#1}}
\newcommand{\masc}{\symbolglyph{♂}}
\newcommand{\fem}{\symbolglyph{♀}}

\newfontfamily{\GlotSubstFont}[Script=Hebrew]{Palemonas MUFI}

\XeTeXinterchartokenstate=1
\newXeTeXintercharclass\GlotSubst

\XeTeXcharclass"02BE=\GlotSubst
\XeTeXcharclass"02BF=\GlotSubst

\XeTeXinterchartoks 0 \GlotSubst = {\begingroup\GlotSubstFont}
\XeTeXinterchartoks 255 \GlotSubst = {\begingroup\GlotSubstFont}
\XeTeXinterchartoks \GlotSubst 0 = {\endgroup}
\XeTeXinterchartoks \GlotSubst 255 = {\endgroup}


\begin{document}

\section*{Introduction}

Hi,

My name is Júda Ronén. I'm a linguist and a native speaker of Hebrew. Although I have done no serious research on \hl{Modern Hebrew}, I think it is a very interesting language, one which I would find \hl{fascinating} to learn had I learnt it as a foreign language. It is often perceived as a \hl{difficult language}\footnote{Actually, there is a self-referential idiom in Hebrew~— \H{עברית קשה שפה} \trans{Ivrit kaša safa}~— which is translatable as ‘Hebrew is a language difficult’, mirroring its ungrammaticality.}, but I believe it is not much more difficult to learn than many other good languages, and with a proper method\footnote{You may think I think my method is a proper one~— I do~— but I'm well aware it doesn't suit everyone. Some people need their inflectional tables laid before them in a systematical way and to learn a great amount of grammar before they approach a real text, some do not, and catch the language better when they encounter real texts right from the beginning. It is for the latter that this course is designed for.} it can be learned not without an effort but with satisfaction and interest.

\emph{Learn Hebrew with Winnie-the-Pooh} is a course for \hl{learning to read Modern Hebrew}. I tried to make it exactly the kind of course I'd like to have for languages I learn. While there are lots of ‘Teach yourself Hebrew in \emph{N} easy lessons’ ***



\subsection*{The text}

***



\subsection*{Orthography, transliteration, transcription and pronunciation}

\grey{(You can \hl{skip} this subsection if you are not interested in learning to read the Hebrew script or how to pronounce the transcription, both are very recommended if you want to use the language, whether for reading other texts or for speaking.)}

Before we begin~— and we will begin soon reading the text~— let's have a look on the Modern Hebrew \hl{writing system}, which is based on the biblical one with some (more or less) systematical differences. Hebrew is written right to left. Its alphabet consists of 26 letters, all of them have consonantal readings and some can act as \emph{matres lectionis}, that is, indicating a vowel.

So here are the \hl{Hebrew letters} along with their \hl{standard transliteration}, which we will use here and is quite standard in Semitic linguistics:\footnote{For the sake of clarity, the forms of \H{כ}, \H{מ}, \H{נ}, \H{פ} and \H{צ} when occurring in final position (i.e.~\H{ך}, \H{ם}, \H{ן}, \H{ף} and \H{ץ}) are indicated by small capitals.

	The names of the letters are (colloquial pronunciation): álef, bet, gímel, dálet, hej, vav, zájin, xet, tet, jud, kaf (-sofit), lámed, mem, nun (-sofit), sámex, ájin, pej (-sofit), cádi (-sofit), kuf, rejš, šin, taf.
}

\begin{center}
\H{
\transliterate{א}{ʾ}
\transliterate{ב}{b}
\transliterate{ג}{g}
\transliterate{ד}{d}
\transliterate{ה}{h}
\transliterate{ו}{w}
\transliterate{ז}{z}
\transliterate{ח}{ḥ}
\transliterate{ט}{ṭ}
\transliterate{י}{j}
\transliterate{כ}{k}
\transliterate{ך}{\sofit{k}}
\transliterate{ל}{l}
\transliterate{מ}{m}
\transliterate{ם}{\sofit{m}}
\transliterate{נ}{n}
\transliterate{ן}{\sofit{n}}
\transliterate{ס}{s}
\transliterate{ע}{ʿ}
\transliterate{פ}{p}
\transliterate{ף}{\sofit{p}}
\transliterate{צ}{ṣ}
\transliterate{ץ}{\sofit{ṣ}}
\transliterate{ק}{q}
\transliterate{ר}{r}
\transliterate{ש}{š}
\transliterate{ת}{t}
}
\end{center}

\noindent
\hl{Peculiarities and regularities of the script will be taught along with the text}, based on the forms we will encounter. Although transcription is used throughout the course, in addition to the original text, the student is encouraged to learn to read the text in its original form by what xe will be taught and by generalizing rules and (ir)regularities on xyr own. For convenience, the first five sentences are fully transliterated; after that, you are expected to compare the original forms with the transcriptions on your own (it shouldn't be very difficult to learn the letters: there are only 26 of them). As you will see, the Hebrew alphabet isn't really well suited for Modern Hebrew, and many of the words have historical motivation for their orthography, rooted in Biblical Hebrew pronunciation.

Modern Hebrew has \hl{five vowels}, which is quite normal cross-linguistically: \footnote{I'm not a phonetician; I'm interested much more in ‘high-level’ aspects of language, such as narrative grammar and macro-syntax. The IPA values are from Wikipedia.} a [\IPA{ä}]}, {e [\IPA{e̞}]}, {i [\IPA{i}]}, {o [\IPA{o̞}]} and {u [\IPA{u}]}.

Winnie-the-Pooh, as all children books, is fully vocalized: the vowels are indicated using diacritics called \transcribe{נִקּוּד}{nikud}. As with the letters, these fit better to older stages of the language, and their use and systematics is often anachronistic.

\begin{center}
\H{
\transliterate{בְ}{Ø/e}
\transliterate{חֱ}{e}
\transliterate{חֲ}{a}
\transliterate{חֳ}{o}
\transliterate{בִ(י)}{i}
\transliterate{בֵ(י)}{e(j)}
\transliterate{בֶ}{e}
\transliterate{בַ}{a}
\transliterate{בָ}{a}
\transliterate{בֹ/בוֹ}{o}
\transliterate{בֻ/בוּ}{u}
}
\end{center}

\noindent
In addition to vowels, \emph{nikud} also differentiates \transcribe{בּ}{b}, \transcribe{כּ}{k} and \transcribe{פּ}{p} from \transcribe{ב}{v}, \transcribe{כ}{x} and \transcribe{פ}{f} and \transcribe{שׁ}{š} from \transcribe{שׂ}{s}, and indicates final \transcribe{־הּ}{-h}.

Now, as for the \hl{transcription} we use, it is quite \hl{straight forward}. Here are the \hl{consonants}, in practical but not pedantic tabular form (greyed letters indicate phonemes which occur only in loanwords):

\begin{center}
\fontspec{Gentium Plus}\footnotesize
\begin{tabular}{llllllllll}
	& \scriptsize Bil. & \scriptsize Lb.-dn. & \scriptsize Dent. & \scriptsize Alv. & \scriptsize Pal.-alv. & \scriptsize Pal. & \scriptsize Vel. & \scriptsize Uvul. & \scriptsize Glot.\\
	\scriptsize Nas. & m & & & n\\
	\scriptsize Stop & p · b & & & t · d & & & k · g & & ʾ\,[ʔ]\\
	\scriptsize Affr. & & & & c\,[t͡s] & \grey{č\,[t͡ʃ]} · \grey{ǧ\,[d͡ʒ]}\\
	\scriptsize Fric. &  & f · v & \grey{þ\,[θ]} · \grey{ð} & s · z & š\,[ʃ] · \grey{ž\,[ʒ]} & & & x\,[χ] · r\,[ʁ̞] & h\\
	\scriptsize Appr. & & & & l & & j & \grey{w}
\end{tabular}
\end{center}

One-lettered prepositions and other proclitic grammemes which are written with no word separation are transcribed \hl{in separation}, in a way similar to Modern English writing. Word-separation here is sensitive to grammaticalization (cf.~English \emph{nevertheless}).

\hl{Stress} is marked by acute accent, unless it is ultimate, which is the most common case and therefore left unmarked.

%Note that \hl{the way people read a written text may vary greatly}. There are some more prestigious (usually archaic) ways to read many words. What I've written down is my own idiolect when I read stories to my children, transcribed as honestly as I can. Note that it is uses slightly more archaic pronunciation than everyday spoken Hebrew, and that it may vary according to the textual situation (e.g., when a character speaks in a less prestigious register, it may show in the way I read and transcribe xyr words). Nevertheless, it is not strict formal Hebrew according to the Academy of the Hebrew Language and may contain personal idiosyncratisms.

Note that the way people \hl{read} a written text (with or without \emph{nikud}) may \hl{vary} greatly, as does the way they use Spoken Hebrew. ***

Now, without further ado, lets' begin…



\newpage

\section*{Winnie-the-Pooh~— the first chapter}

\newparagraph{}

\begin{sentence}
	{הִנֵּה אֱדוּאַרְד דֹּב, יוֹרֵד בַּמַּדְרֵגוֹת, טַאח, טַאח, טַאח, הוּא נֶחְבָּט בְּעָרְפּוֹ, נִגְרָר אַחֲרֵי כְּרִיסְטוֹפֶר רוֹבִּין.}
	{Hine Eduard Dov, jored ba madregot, tax, tax, tax, hu nexbat be orpo, nigrar axarej Krístofer Róbin.}
	{Here is Edward Bear, coming downstairs now, bump, bump, bump, on the back of his head, behind Christopher Robin.}
\end{sentence}

\begin{sentence}
	{זוֹהִי, עַד כַּמָּה שֶׁהוּא יוֹדֵעַ, הַדֶּרֶך הַיְּחִידָה לָרֶדֶת בַּמַּדְרֵגוֹת, אֶלָּא שֶׁלִּפְעָמִים נִדְמֶה לוֹ שֶׁחַיֶּבֶת לִהְיוֹת דֶּרֶך אַחֶרֶת, לוּ רַק הָיָה יָכוֹל לְהַפְסִיק לְהֵחָבֵט לְרֶגַע וְלַחְשֹׁב עַל זֶה.}
	{Zóhi, ad kama še hu jodéa, ha dérex ha jexida larédet ba madregot, ela še lif'amim nidme lo še xajévet lihjot dérex axéret, lu rak haja jaxol lehafsik lehexavet le réga ve laxšov al ze.}
	{}
\end{sentence}

\begin{sentence}
	{וְאָז נִדְמֶה לוֹ שֶׁאוּלַי בְּעֶצֶם אֵין.}
	{Ve az nidme lo še ulaj be écem ejn.}
	{}
\end{sentence}

\begin{sentence}
	{כָּך אוֹ כָּך, הִנֵּה הוּא כְּבָר לְמַטָּה, וּמְחַכֶּה שֶׁיַּצִּיגוּ אוֹתוֹ בִּפְנֵיכֶם. וִינִי־הַפּוּ.}
	{Kax o kax, hine hu kvar lemáta, u mxake še jacígu oto bifnejxem. Vini-ha-Pu.}
	{}
\end{sentence}

\end{document}
